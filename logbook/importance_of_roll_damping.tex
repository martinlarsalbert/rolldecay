\section{Literature review on the importance of roll damping}

\subsection{Parametric roll}

\emph{"Parametric roll (a shortening of the formal term "parametric roll resonance") is a dynamic stability phenomena in which an amplification of roll motion is caused by periodic variation of transverse stability in waves. The phenomenon of parametric roll is predominantly observed in head, following, bow and stern-quartering seas when the ship's encounter frequency is approximately twice that of the ship's roll natural frequency and the ship's roll damping is insufficient to dissipate additional energy (accumulated because of parametric resonance)."} \parencite{imo_second_nodate}

\subsection{Influence of roll damping}
\subsubsection{}
\emph{"When a ship rolls in calm water after being disturbed, the roll amplitudes decrease successively due to roll damping, see figure 4. A rolling ship generates waves and eddies, and experiences viscous drag. All of these processes contribute to roll damping. Roll damping may play a critical role in the development of parametric roll resonance. \textbf{If the "loss" of energy per cycle caused by damping is more than the energy "gain" caused by the changing stability in longitudinal seas, the roll angles will not increase and the parametric resonance will not develop.} Once the energy "gain" per cycle is more than the energy "loss" due to damping, the amplitude of the parametric roll starts to grow."} \parencite{imo_second_nodate}

\subsubsection{}
\emph{"An accurate description of the roll damping is crucial for prediction of critical roll responses such as parametric roll. Sufficient damping can be the difference between a severe roll response and a hardly noticeable response."} \parencite{soder_ikeda_2019}
